\documentclass[a4paper,12pt]{article}
\usepackage{graphicx}
\usepackage{amsmath}
\usepackage{geometry}
\usepackage{longtable}
\usepackage{booktabs}

\title{Data Extraction, Preprocessing and Clustering Report:\\ EcoWheataly – Durum Wheat Dataset}
\author{EcoWheataly Team}
\date{\today}

\begin{document}

\maketitle

\section{Objective}
This script is designed to extract, merge, and preprocess agricultural and economic data from the EcoWheataly JSON database for durum wheat farms in Italy. The goal is to generate a clean and structured panel dataset suitable for machine learning tasks, particularly clustering and predictive modeling.

\section{Input}
\begin{itemize}
    \item \texttt{ecowheataly\_database.json} – a nested JSON file containing farm-level data by year and crop.
    \item Analysis range: 2008 to 2022.
\end{itemize}

\section{Structure of the Database}
Each farm entry contains:
\begin{itemize}
    \item General info (e.g., \texttt{farm\_acreage})
    \item Crop-specific data under species key (\texttt{durum\_wheat}), such as:
    \begin{itemize}
        \item Production, value, area, cost variables
        \item Fertilizers (categorized by type)
        \item Phytosanitary products (categorized by toxicity and type)
    \end{itemize}
\end{itemize}

\section{Extraction Workflow}
\subsection{Part 1: General Variables and Fertilizers}
Extracts key indicators for each farm-year:
\begin{itemize}
    \item Wheat production: \texttt{produced\_quantity}, \texttt{PLV}, \texttt{crop\_acreage}, etc.
    \item Economic data: \texttt{fert\_costs}, \texttt{phyto\_costs}, \texttt{human\_costs}, etc.
    \item Fertilizer details: area treated, amount per hectare, unit cost, nitrogen/phosphorus/potassium content, divided by type.
\end{itemize}

\subsection{Part 2: Fertilizers by Type}
For each fertilizer type (\texttt{Mineral}, \texttt{OrganoMineral}, etc.), the following variables are extracted:
\begin{itemize}
    \item \texttt{fert\_area}, \texttt{whole\_qt\_ha}, \texttt{unit\_cost}, \texttt{distribuited\_value}
    \item Nutrients per hectare: \texttt{nitrogen\_ha}, \texttt{phosphorus\_ha}, \texttt{potassium\_ha}
\end{itemize}

\subsection{Part 3: Phytosanitary Products}
For each phytosanitary type (\texttt{Herbicide}, \texttt{Insecticide}, \texttt{Fungicide}), the script:
\begin{itemize}
    \item Extracts distribution quantity by toxicity class (0–4)
    \item Aggregates quantity per hectare for each class
    \item Stores data in a 3D array (\texttt{Phyto}) and flattens to a 2D DataFrame
\end{itemize}

\section{Data Merging and Cleaning}
\begin{itemize}
    \item All extracted data frames (\texttt{df1}, \texttt{df2}, \texttt{df3}) are merged on \texttt{year} and \texttt{farm code}
    \item Missing phytosanitary data are filled with zeros (assumed not used)
    \item NaNs in other sections are handled and flagged for filtering
    \item Diagnostic statistics per year and farm are printed
\end{itemize}

\section{Visualization}
\begin{itemize}
    \item Boxplots and trend plots of phytosanitary product usage by toxicity class
    \item Indicators such as zero-ratio for selected variables (e.g., \texttt{fert\_costs})
\end{itemize}

\section{Output}
\begin{itemize}
    \item Cleaned and flattened DataFrame saved as \texttt{flat\_df2.csv}
    \item Ready for clustering and machine learning pipelines
\end{itemize}

    \section{Preliminary Analysis of Results}

After constructing the full panel dataset, an initial exploration was conducted to assess data completeness and potential signals for clustering.

\subsection{Missing Data Overview}
For each year, the script reports:
\begin{itemize}
    \item Total number of farm-level observations
    \item Number of complete records with all key variables available
    \item Number of distinct farms producing \texttt{durum\_wheat}
\end{itemize}

This provides insight into the evolution of data coverage and production activity over time.

\subsection{Zero-Ratio Analysis}
For selected economic and input-related variables such as:
\begin{itemize}
    \item \texttt{PLV}, \texttt{fert\_costs}, \texttt{phyto\_costs}
    \item \texttt{Mineral\_distribuited\_value}, \texttt{Mineral\_nitrogen\_ha}
\end{itemize}
the share of zero values is computed. High zero-ratio values may indicate non-use, data quality issues, or real heterogeneity in input intensity.

\subsection{Phytosanitary Product Trends}
Toxicity-class-specific quantities are plotted over time for each phytosanitary category:
\begin{itemize}
    \item Boxplots show distribution of \texttt{log(1 + quantity\_ha)} per class and year.
    \item Line plots show mean quantities per class across years.
\end{itemize}

This visualization allows assessment of both intensity and spread of use by type and toxicity level.

\subsection{Data Cleaning and Final Dataset}
After filtering out incomplete observations, the cleaned dataset is stored in \texttt{flat\_df2.csv}. This dataset contains:
\begin{itemize}
    \item Time-consistent and farm-aligned information
    \item Explicit zeros for missing phytosanitary use (assumed non-use)
    \item Numerical consistency checks (e.g., removal of infinite values)
\end{itemize}

This forms the basis for downstream tasks such as clustering of farm strategies or predictive modeling.




\end{document}
